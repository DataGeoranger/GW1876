%% LyX 2.2.3 created this file.  For more info, see http://www.lyx.org/.
%% Do not edit unless you really know what you are doing.
\documentclass[english]{article}
\usepackage[T1]{fontenc}
\usepackage[latin9]{inputenc}
\usepackage{geometry}
\geometry{verbose,tmargin=1.5cm,bmargin=1.5cm,lmargin=1.5cm,rmargin=1.5cm}
\begin{document}

\title{Applied Groundwater Model Calibration and Uncertainty Analysis Curriculum }

\author{Michael N. Fienen, Jeremy T. White and Randall J. Hunt }

\date{April 23-27, 2018}

\maketitle

\textbf{MONDAY}
\begin{enumerate}

\item \textbf{Introductions of students/instructors {[}MNF{]}.} Goals for
the week and framing 
\begin{enumerate}
\item The mechanics and theory 
\item Learning by doing 
\item Please speak up! Everyone learns from discussion 
\item Work in pairs 
\item Python, GUIs, and all that 
\end{enumerate}

\item \textbf{Shortish Historical Background \#1 and Introductory material}
{[}RJH{]}. What is \textquotedblleft inverse\textquotedblright ? How
did we get here? The concepts of model usage, forecasting, how type
of forecast influences uncertainty, learning, uncertainty analysis
and the role of calibration. PE is a tool to an endpoint - not inviolate.
A perfect fit for Stepwise modeling: a tool for automated trial-and-error,
and tweak at end. MAIN COURSE POINT: There is no model of a part of
the world, it is ALWAYS a simplification of reality and our job is
to tune the simplifications for the forecasts of interest

\item \textbf{Logistics and airing of the IT grievances }{[}MNF{]}: 
\begin{enumerate}
\item Did everyone get the software installed? 
\item Pull the class from github and make a copy 
\item Brief Git tutorial  
\end{enumerate}

\item \textbf{intro\_to\_regression} {[}MNF{]} Linear regression overview-{}-{}-illustration
with fitting a polynomial to noisy data including the role of SSE
(response curve), degrees of freedom, the gradient of that surface
(Jacobian) overfitting (give fish) MAIN COURSE POINT: Not magic, automated
trial-and-error (Maths!!!). The models will get more involved, need
a more powerful tool (PEST++) 

\item \textbf{freyberg\_model\_intro} {[}MNF{]}: Introduce Freyberg model
outlining the role of forecasts, the increase in parameterization.
Introduce forecasts of interest: head, sw-gw flux, travel time

\item \textbf{freyberg\_trialerror} {[}JTW{]} Messing around with a pre-setup
version of the Freyberg model using trial and error to explore fitting
a couple parameters looking at both calibration and forecast implications.
Find parameter values that best fit the calibration heads, and also
parameters that fit the travel time forecast 

\item \textbf{freyberg\_pest\_setup} {[}RJH{]}: Mechanics and Details on
PEST++, how to run, control file (refer to annotated print out), model
run command (mention importance of a forward run script), TPL and
INS files and controlling MODFLOW; PESTCHEK: use TEMPCHEK and INSCHEK
to write simple input and output for the freyberg single K model.
Run PESTCHEK and then run PEST++. Have control file made for simple
1 par case, specify par and obs names. Show how forecasts can be observations
in control file. 

\item \textbf{freyberg\_pestcheck\_trouble} {[}RJH{]} fix a troubled PEST
control file (trouble.pst), see an error that is not caught but by
PESTCHEK but is caught when PEST++ is run, make them uncomment a PEST++
comment line to get their forecast back in the rec file 

\item \textbf{Why we model} {[}JTW{]} Outlining forecasts to be made in
first model runs and relate to why we model, and why looking at calibration
is often not enough for good forecasts. Underscore MAIN COURSE POINT:
we are focused on forecasts- model purpose. What does parameter estimation
teach us about the model and its forecasts? Tie back to previous example/activity 
\begin{enumerate}
\item Models are built because of uncertainty Quantities of interest(QOIs)/forecasts/predictions
can\textquoteright t be observed To properly \textquotedblleft model\textquotedblright ,
we should then have a \textquotedblleft learning\textquotedblright{}
framework: Learning == reduction in uncertainty Model analyses combine
information our knowledge about the system Observations of the system
(deterministic) parameter estimation is a small piece of modeling
analyses Part of the \textquotedblleft learning\textquotedblright -from-observations
process But parameter estimation itself does not help us learn about
QOIs Before \textquotedblleft calibration\textquotedblright{} the
QOIs a have a single \textquotedblleft value\textquotedblright{} After
\textquotedblleft calibration\textquotedblright{} the QOIs have a
(new) single \textquotedblleft value\textquotedblright{} Tuesday 
\end{enumerate}

\item \textbf{introduction\_to\_GLM }{[}MNF{]} Derivation of the Gauss-Levenberg-Marquardt
Algorithm part I. The maths

\textbf{TUESDAY}

\item \textbf{freyberg\_k} {[}RJH{]} Review the PEST++ control file...refer
to annotated print out (provided) (need to decide balance of details
day 1 - day 2 ). Use PEST++ to calibrate with only a single hydraulic
conductivity (K) zone and head observations: we provide fully hooked-up
MODFLOW files. We will walk through the pest setup components (TPL,
INS, PST). Look at .rec file for parameter and forecast uncertainty.
Explore how parameter bounds and observation weights influence the
resulting forecast uncertainty. Reintroduce parameter estimation conceptually
(tying back to regression, and highlighting forecasts). Parameter
estimation is just a means to extract info from obs and store in parameters...we
don\textquoteright t know how that information flowed until uncertainty
analysis. GUI equivalent - hit the wall. 

\item \textbf{Short Interlude - what does a response surface tell us?} {[}RJH{]}
Quick and dirty activity, intro on response function, setup grid-based
sampling and use sweep util to gen runs - allow them to specify their
own weights for obs and PI - should give different response surface
- nonuniqueness!!! 

\item \textbf{xsec\_response\_surface\_GLM} {[}MNF{]} Derivation of the Gauss-Levenberg-Marquardt
Algorithm part II: response surface. Discussion of the
algorithm in terms of the response surface, GLM, Use previous response
surface to show steepest vs newton direction and how lambda works
and why lambda is a regularization device. 

\item \textbf{freyberg\_k\_and\_r} {[}JTW{]} use PEST++ to calibrate both
K and recharge (R) with head observations. Have premade files for
this. DIY - have students add recharge parameter. With so many head
targets why isn\textquoteright t this overdetermined? Why did forecast
uncertainty go up? 

\item \textbf{freyberg\_k\_and\_r\_response\_surface} {[}MNF{]} Correlation
and non-uniqueness in regression-{}-Response surface illustration
(using the Freyberg model): the need for diverse data. Discuss how
the flux observation made a more unique solution and what this additional
information does to uncertainty \textemdash{} going toward a good
compromise (MAIN COURSE POINT: observations don\textquoteright t cost
us runs; good practice put in as many observation types as possible).
Show updated response surface with lovely bullseye replacing banana
canyon 

\item \textbf{freyberg\_k\_and\_r\_fluxobs} {[}RJH{]} use PEST++ to calibrate
K and R with a flux observation and heads (we provide all files).
MAIN COURSE POINT: Burning silicon can\textquoteright t make up for
poorly thought out problems 

\item \textbf{Adding a flux target in GWV GUI digression} {[}RJH{]} example
of PEST++ file generation, GUI only as starting point. Include fluxobs
unfix \textquotedblleft rch\_1\textquotedblright{} and look at how
including future climate uncertainty impacts forecast uncertainty.
Recharge is most sensitive parameter for calibration, makes sense
it is important for forecasting the future. Fixing it implies we know
the future perfectly. 

\item \textbf{freyberg\_reweighting} {[}MNF{]} The role of weighting and
how to balance objective function. Doing it at the beginning - not
the end (pyEMU, pie charts, weight-o-matic). Tie back to previous
response surface activity. 

\item \textbf{intro\_to\_sweep} {[}JTW{]} Show that response surface was
not a built-in PEST utility. Sweep is general and allows you to setup
whatever parameter sweep you want. Demo with a single parameter set
csv. Demo YAMR - have students rerun k+r calibration, manually starting
master and workers.

\item \textbf{freyberg\_k\_and\_r\_mc} {[}RJH{]} run MC on the K and R model
but drawing from prior \textemdash{} talk about nonuniqueness and
rejection sampling (GLUE). MAIN COURSE POINT: MC is a stakeholder
friendly, easy-to-understand uncertainty technique, but incurs a huge
computational burden for real problems. How many realizations are
\textquotedblleft acceptable\textquotedblright{} in terms of phi?
How does the best \textquotedblleft phi\textquotedblright{} compare
to the calibrated phi? Including Monte Carlo analysis-{}-{}-we can\textquoteright t
calibrate future recharge, but we can quantify how much it matters.
The posterior for rch\_1 is the same as the prior - why is that? 

\textbf{WEDNESDAY}

\item \textbf{freyberg\_zones} {[}MNF{]} Set up zones for K and calibrate
with PEST++. Using a given zonation and a zone-burner python script. 
\item \textbf{freyberg\_global\_sensitivity} {[}RJH{]} How can we extend
diagnosis with SVD -{}- which parameters are informed by the data?
Scale, offset, tying, fixing. What matters? 

\item \textbf{intro\_to\_bayes} {[}MNF{]} Really, calibration is a small
component of Bayes\textquoteright{} Rule (and model usage)\textemdash parameters
are conditional on information (types and amounts of information)
and model structure (\textquotedblleft calibrated\textquotedblright{}
parameters aren\textquoteright t transferable)- if forecast depends
on those parameters, then we learn about the forecast. Put another
way, discuss the process in terms of learning as a reduction in uncertainty. 

\item \textbf{freyberg\_zones\_DIY\_mc} {[}JTW{]} run both FOSM and MC on
the zoned freyberg model - Compare FOSM and MC for both performance
and results. Also explore identifiability 

\item \textbf{intro\_to\_FOSM} {[}JTW{]} FOSM as an alternative to MC (rejection
sampling). Implications of \textquotedblleft implied\textquotedblright{}
calibration, linearization, speed. MAIN COURSE POINT: MC is often
easy to communicate results, but FOSM is much cheaper. How can FOSM
be completed before history matching? It\textquoteright s magic!

\item \textbf{intro\_to\_dataworth} {[}MNF{]} FOSM- huh, good god y\textquoteright all.
What is it good for? Absolutely nothing! Quantification of prior and
posterior forecast uncertainty (the value of calibration), but more
importantly, data worth considerations. Linearity assumption may not
be a good approximation in all cases, but still useful to eval the
value of different calibration, different data, etc. What parameters
are important to uncertainty in forecasts (modern alternative to \textquotedblleft sensitivity
analysis\textquotedblright{} (USGS-type)) PEST lingo = PREDUNC/GENLINPRED 

\item \textbf{freyberg\_zones\_dw} {[}MNF{]} Data worth with the zoned model.
Next best observation location to measure to reduce forecast uncertainty.
We provide an alternate zonation option. Discuss the ramifications
of various zonations on both forecasts The value of flexibility in
parameter fields with special emphasis on avoiding overfitting-{}-we
fit to extract maximum info from the observations while also capturing
parameter contributions to forecast uncertainty.

\item \textbf{GUI\_demo\_with\_GW\_Vistas} {[}RJH{]} Overview of how to
use GW Vistas for setting up, running, and postprocessing zoned model. 

\item \textbf{intro\_to\_geostatistics} {[}MNF{]} Exploring the idea of
a model of spatial correlation (e.g. covariance matrix). Variograms
as a basis for interpolation, the use of factors, \textquotedblleft spatially
weighted averaging\textquotedblright , all this wrapped up in Kriging.
Function for creation and viz. In pyemu. Namecheck sGems, Surfer,
TPRogs

\textbf{THURSDAY}

\item \textbf{intro\_to\_pyemu} {[}JTW/MNF{]} The mechanics behind the beast.
Control file handling and manipulation, matrix/cov/jco handling. Res
handling, etc. Setting up the interface with flopy models. Pilot points,
geostats, prior info construction, starting local slaves. FOSM (schur
and error variance), MC/ensemble handling, various output setup and
apply routines (hob, hyd, list file, hds).

\item \textbf{freyberg\_pilot\_points\_1} {[}JTW{]} Create a pilot point
network and run pestpp. YAMR for parallel processing on laptops. Cover
the implementation of setting a pp network and setting geostats factors.
No regularization...way over fit Perhaps need a brief intro to pilot
points to explain the concept, history, importance of representative
starting values, and high parameter flexibility does not equal high
complexity

\item \textbf{intro\_to\_singular\_value\_decomposition} {[}MNF{]} an aside
backing up the theory and practice of SVD and a little on SVDA because
now that we are moving to PP land we need a crutch. Concepts of relinearization
- think of a boss analogy. Make sure to discuss fundamental subspaces.
Give chain of PEST developments that were caused by pilot points.
Also TSVD and \textquotedblleft super parameters\textquotedblright{}
Need to talk about TSVD as a regularization device - the MAP implementation
of GLM Need thought experiments with MAXSING and EIGTHRESH - what
happens conceptually with these...

\item \textbf{freyberg\_sensitivity\_identifiability} {[}RJH{]} How can
we extend diagnosis with SVD -{}- which parameters are informed by
the data? What matters? What\textquoteright s the value of ID vs.
Sensitivity. Correlation issues. Compare the two. 

\item \textbf{intro\_to\_regularization} {[}RJH{]} introduction to the concepts,
theory and implementation of Tikhonov regularization. Need to talk
about TSVD as a regularization - they will see this in the next exercise
Bringing in geostats concepts. Need to talk about constrained optimization
with Lagrange multiplier? 

\item \textbf{freyberg\_pilot\_points\_2} {[}RJH{]} Add prior information
and switch to regularization mode. What happens now? Talk about overfitting:
parameters and forecasts. 

\item \textbf{freyberg\_pp\_GUI} {[}RJH{]} GUI digression for creating PP
network with regularization; debug file of steps. Maybe some discussion
of other approaches 

\item \textbf{parallel\_processing }{[}JTW/MNF{]} As an aside, while running
the above, show YAMR manually, maybe HTCondor too. 

\textbf{FRIDAY}

\item \textbf{freyberg\_pp\_dw} {[}JTW{]} Revisiting FOSM and data worth
with the pilot point model-{}-fare thee well artifacts. Diagonal prior
was based on bounds. Have a full prior built for use also - plot for
demystification. Why does uncertainty go up when a non-diagonal prior
is used? 

\item \textbf{freyberg\_pp\_monte\_carlo }and \textbf{freyberg\_pp\_advanced\_monte\_carlo}
{[}JTW{]} Monte Carlo with various sampling strategies to increase
number of realization retained. Three strategies are: 0) sample Schur-derived
posterior, 1) vanilla style with the prior full covariance. 

\item \textbf{Discussion} {[}ALL{]} what other model inputs could be parameters
(what is not known perfectly)? The importance of parameterization:
\textquotedblleft If it\textquoteright s not a parameter, it must
be known perfectly\textquotedblright . What are other sources of uncertainty?
Future stresses as parameters? 

\item \textbf{freyberg\_grid\_ies} {[}JTW{]} iterative ensemble smoothers.
Allow all BCs, future stresses, etc. to be parameters and explore
uncertainty with FOSM. Use the pyemu example freyberg notebook for
this. JTW
\end{enumerate}

\end{document}
