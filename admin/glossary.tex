\documentclass[english]{article}
\usepackage[T1]{fontenc}
\usepackage[latin9]{inputenc}
\usepackage{geometry}
\geometry{verbose,tmargin=1.5cm,bmargin=1.5cm,lmargin=1.5cm,rmargin=1.5cm}

\begin{document}
\title{Glossary of important terms for GW1876}
\maketitle

\begin{description}
\item [Parameters] Variable input values for models, typically representing system properties and forcings. Values to be estimated in the history matching process. Typically identified as $k$, $p$, or $x$ ($\mathbf{k}$, $\mathbf{p}$ or $\mathbf{x}$ for multiple parameters in a vector).
\item [Observation] Measured system state values. These values are used to compare with model outputs collocated in space and time. The term is often used to mean \emph{both} field measurements and outputs from the model. When referring to a measured value, observations are typically identified by the variables $y$ or $o$  ($\mathbf{y}$ or $\mathbf{o}$ for multiple parameters in a vector)
\item [Modeled Equivalent] A modeled value collocated in time and space with an observation. There are various ways to identify a single or multiple modeled equivalent values (and, to make things confusing, they are often \emph{also} called ``observations"!)  \newline{}
\textbf{Single values} 
\begin{enumerate}
\item $f\left(x\right)$
\item $X\left(\beta\right)$
\item $M\left(p\right)$
\end{enumerate}
\textbf{Multiple values}
\begin{enumerate}
\item $\mathbf{X}\beta$
\item $\mathbf{M}\mathbf{p}$
\item $\mathbf{NOBS}$ Number of observations/simulated equivalents in the inverse model setup
\item $\mathbf{NPAR}$ Number of adjustable input parameters in the inverse model setup

\end{enumerate}
\item [Forecasts] Model outputs for which field observations are not available. Typically these values are simulated under an uncertain future condition.
\item [Phi] Objective function, defined as the weighted sum of squares of residuals. Phi (aka $\mathbf{\Phi}$) is typically calculated as
\begin{equation}
\begin{array}{ccc}
 \Phi=\sum_{i=1}^{n}\left(\frac{y_{i}-f\left(x_{i}\right)}{w_{i}}\right)^{2} & or & \Phi=\left(\mathbf{y}-\mathbf{Jx}\right)^{T}\mathbf{Q}^{-1}\left(\mathbf{y}-\mathbf{Jx}\right)
 \end{array}
\end{equation}
\item [Residuals] The difference between observation values and modeled equivalents $r_i=y_i-f\left(x_i\right)$
\item [Sensitivity] The incremental change of an observation (modeled equivalent, actually) due to an incremental change in a parameter. Typically expressed as a finite-difference approximation of a partial derivative: $\frac{\partial y}{\partial x}$
\item [Jacobian Matrix] A matrix of the sensitivity of all observations in an inverse model to all parameters. This is often shown as a matrix by various names $\mathbf{X}$, $\mathbf{J}$, or $\mathbf{H}$. Each element of the matrix is a single sensitivity value  $\frac{\partial y_i}{\partial x_j}$ for $i\in NOBS$, $j \in NPAR$
\item [Regularization] A preferred condition pertaining to parameters, the deviation from which, elicits a penalty added to the objective function. This serves as a balance 
\item [PHIMLIM] A PEST input parameter the governs the strength with which regularization is applied to the objective function. I high value of PHIMLIM indicates a strong penalty for deviation from preferred parameter conditions as 
\item [FOSM]
\item [Gaussian (multivariate)]
\item [Weight] 
\item [Weight Covariance matrix (correlation matrix)]
\item [Parametric uncertainty]
\item [Measurement noise]
\item [Structural (model) error]
\item [Monte Carlo Ensemble]
\item [Bayes' Theorem]
\item [Posterior (multivariate distribution)]
\item [Schur Complement]
\item [Prior (multivariate distribution)]
\item [Likelihood (multivariate distribution)] 

\end{description}

\end{document}