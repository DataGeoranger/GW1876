%% LyX 2.2.3 created this file.  For more info, see http://www.lyx.org/.
%% Do not edit unless you really know what you are doing.
\documentclass[english]{article}
\usepackage[T1]{fontenc}
\usepackage[latin9]{inputenc}
\usepackage{geometry}

\usepackage{xcolor}
\geometry{verbose,tmargin=1.5cm,bmargin=1.5cm,lmargin=1.5cm,rmargin=1.5cm}
\begin{document}

\title{Applied Groundwater Model Calibration and Uncertainty Analysis Curriculum }

\author{Jeremy T. White, Michael N. Fienen,  and Randall J. Hunt }

\date{Feb 2020}

\maketitle

\textbf{MONDAY}
\begin{enumerate}

\item \textbf{Introductions of students/instructors} Goals for
the week and framing 
\begin{enumerate}
\item The mechanics and theory 
\item Learning by doing 
\item Please speak up! Everyone learns from discussion 
\item Work in pairs 
\item Python, GUIs, and all that 
\end{enumerate}

\item \textbf{Logistics and airing of the IT grievances }: 
\begin{enumerate}
\item Did everyone get the software installed? 
\item Pull the class from github and make a copy 
\item Brief Git tutorial (including modeling workflows with Git)
\end{enumerate}

\item \textbf{RRR} Long boring talk about modeling.

\item \textbf{Regression}

\item \textbf{Demo with EAA model} run thru Edwards Aquifer modeling workflow?



\item \textbf{Freyberg model} Introduce the enhanced Freyberg model.
Show "truth" observations, water budget, etc

\item \textbf{Build a transient model notebook}. flopy/python review

\textbf{TUESDAY}

\item \textbf{Geostatistics, the Prior and pilot points} Exploring the idea of
a model of spatial correlation (e.g. covariance matrix). Variograms
as a basis for interpolation, the use of factors, \textquotedblleft spatially
weighted averaging\textquotedblright , all this wrapped up in Kriging.
Function for creation and viz. Spectral simulation for unconditional simulation.
What are pilot points?  \textcolor{red}{probably need a notebook or two here}

\item \textbf{Connecting a model to PEST++} GUI Style - how can you connect a model to PEST with a GUI?

\item \textbf{Intro to pyemu} {\color{red} need an excersize here}

\item \textbf{Setup pest interface notebook} "what just happened!?".  \textcolor{red}{Excersize: experiment with the 
PstFromFlopy class to setup different pars and obs. competition: who can setup the most parameters?}

\item \textbf{Bayes Theorem and Inference}

\item \textbf{Prior Monte Carlo}. Do this before even setting obsvals or weights.
mechanics of parallel run mgmt - including giveup setting, pestpp-swp.  The safety of Prior Monte Carlo.
Learning from Prior Monte Carlo

\textbf{WEDNESDAY}

\item \textbf{GLM} Derivation of the Gauss-Levenberg-Marquardt
Algorithm part I. The maths.  Vector and ensemble/Monte Carlo forms.  Regularized GLM. 
The dangers of parameter adjustments...  

\item \textbf{ WTF is up with transient modeling?}. Discussion about the allure and pitfalls
of transient modeling, esp for history-matching.

\item \textbf{Processing obs and setting weights notebook}.  Discussion about observation processing



\item \textbf{PESTPP-GLM part 1 notebook: fill a base Jacobian}. Start this then cover FOSM theory

\item \textbf{FOSM theory (including ident and sens) and dataworth} FOSM as an alternative to MC (rejection
sampling). PEST lingo = PREDUNC/GENLINPRED.  Posterior residual-based weight adjust and implications.

\item \textbf{Dataworth notebook} 

\item \textbf{PESTPP-GLM part 2 notebook}. Actual parameter adjustments (scary!) and 
posterior parameter and forecast estimation. {\textcolor{red} change the forecasts and do more dataworth}

\item \textbf{GUI\_demo\_with\_GW\_Vistas} Overview of how to
use GW Vistas for setting up, running, and postprocessing

\textbf{THURSDAY}

\item \textbf{Ensemble-based modeling analyses}. Considerations for using ensembles.
parameter statistical moment changes instead of "sensitivity analysis".  Visualizing 
obs vs sim. more comprehensive pest interface.  

\item \textbf{PESTPP-IES notebook part 1}

\item \textbf{Localization and PESTPP-IES notebook part 2}

\item \textbf{Prior-data conflict - returning to Prior Monte Carlo} Bias-variance tradeoff. Pontification around
the obession with a good fit vs the reality flawed models and noisey data.  Underfitting as a 
strategy to avoid bias.

\item \textbf{PESTPP-IES with PDC activated}

\item \textbf{Class Project: ensemble-based scenario modeling} Create a 
scenario model, setup a pest interface and run prior or posterior Monte Carlo (pestpp-swp)

\textbf{FRIDAY}

\item \textbf{Mgmt opt under uncertainty}

\item \textbf{PESTPP-OPT notebook}

\item \textbf{Discussion and wrap up}

\end{enumerate}

\end{document}
